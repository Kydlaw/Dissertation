\chapter*{Remerciements}

Cette partie est le point final de ces travaux et s'avère également être la dernière page de ce chapitre personnel.
Comme toutes les bonnes aventures, elle n'aurait bien sûr pas été aussi plaisante sans de nombreux compagnons.
Je ne remercierai jamais assez chacun d'eux.

Tout d'abord, mon encadrement de thèse, et en particulier mon directeur et ma directrice.
Frédérick, pour son enthousiame permanent et son incroyable ambition pour ce que je créais.
Nos conversation ont été nombreuses et pourtant j'ai le sentiment qu'il n'y en a pas eu assez,
tant elles ont été productive intellectuellement.
Je peux heureusement me consoler en sachant qu'il y en aura d'autres.\\
Andrea qui a apporté une couleur toute personnelle à ces travaux.
Tout d'abord par son accueil chaleureux lors de mon temps aux États-Unis, ce qui m'a permit d'aborder ces travaux avec un autre regard.
Il va sans dire que ses contributions ne s'arrêtent pas là.
Elle a été un inestimable soutient me supporter et m'accompagner dans mes raisonnements tout au long de ce projet, et pour cela je lui suis très reconnaissant.
J'espère qu'un jour nos chemins se recroiseront de nouveau.

Bien évidemment vient ensuite Aurélie, qui a su marquer ces travaux par son intelligence
mais aussi sa bienveillance et son attention. Tu as su être disponible quand j'avais besoin
de toi et m'aider dans les moments les plus difficiles, et pour cela, je te suis incroyablement reconnaissant.
Tout comme pour Fred, nos innombrables (et parfois longues) conversations qui ont su alimenter nos réflexions respectives.
J'espère que l'on trouvera du temps pour en avoir d'autres.

Je tiens également à remercier Tina Comes et Valentina Dragos d'avoir accepté de relire en détail le présent manuscrit.
Je dois reconnaître leurs patience pour affronter autant de contenu aussi rigoureusement, et les remercier pour leurs nombreuses commentaires perspicaces qui ont su améliorer substantiellement ce manuscrit.

Je remercie également François Charoy, d'avoir habillement présidé le jury de ma soutenance et pour ses remarques profondes sur mes travaux.

En tant qu'invités, j'ai bien évidemment été ravi de revoir Prasenjit Mitra et Jess Kropczynski à cette occasion, en gage de leurs bienveillance sur mes travaux lors de mon passage aux États-Unis.

Caroline Rizza, pour sa gestion du projet MACIV, ainsi que tous les autres acteurs du projet, qui ont permis les nombreuses rencontres et les nombreux retours dont ce travail à amplement bénéficié.

Parmi tous mes compagnons dans cette aventure, certains ont su faire de ce long voyage, un voyage moins long.
Oversea, I obviously think about my housemates, Chloe, Laura and Adam, Erinn, Ryan but also Jenny, Nasim, Connor and Guillermo.
It was good having you around and I definitely miss spending time with you around State College.

- Audrey et Guillaume
- Aurelie et Robin
- Eva
- Raphael
- Robin
- Manon
- Sina

Vous avez définitivement donné une teinte bien plus colorée à cette expérience, au rythme
des nombreuses session de jeux et des nombreux autres moments que l'on a pu échanger ensemble.
Les nombreux moments que l'on a passé ensemble (café etc.)

Mes parents, ma sœur et ma famille en general qui m'ont supporte et encouragé tout au
long de ce voyage vers ces contrés bien lointaines.
Cette réussite .

%%% Local Variables:
%%% mode: latex
%%% TeX-master: "../ma-these"
%%% End:
