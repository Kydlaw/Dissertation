\chapter{Literature review}

%% 
Fil rouge de la LR : Pourquoi - Métier / Quoi / Comment - Technique
Tableau a double entree Metier/Techniques qui met en evidence le manque d'adequation pour repondre au quoi
La colonne metier est repondue a la fin du premier chapitre contribution . Mettre en avant ce manque dans ce chapitre et revenir a la fin du chapitre suivant pour completer le tableau
%%

The aim of the literature review is to highlight the gaps in the literature around my problematic.
It is organized in order to successively narrow the scientific challenges around the sub-problematics identified earlier.
1. What? Crisis situation models that we can implement using sociel media data
2. How? Existing systems to process social media data. Existing systems to implement crisis situation models using social media data
3. Why? What is the reason to implement these models in first place? What is the context in which these systems exist? Operators' needs of information and how do they fit in the picture?

%% Role of social media in crisis situation models
- What is a model? How do we model? (probably going to be an appendix)
- Existing Crisis situation models
- Crisis situation models that take into account the available data on socia media.

=> Point of view model != user of such systems
-> Need to identify user's needs ("business problem")

\section{Systèmes d'implémentation de modèles de situation de crise}
- Implémentation automatique de modèles de situation de crise
- Design de la requête (automatic AND crisis (ontology OR model))
- Implémentation automatique de modèles de situation de crise à partir de données issues de medias sociaux
- Design de la requête (automatic AND crisis (ontology OR model) AND social media)

\section{Social media processing systems}
- Informations à l'échelle du message complet (métadonnées + texte)
- Informations à l'échelle du message textuel
- Informations à l'échelle des mots du message.

\section{Conclusion de la LR et manques identifiées}
1. Pas ou peu d'analyse du besoin des personnes qui utilisent les médias sociaux en situation de crise. Des classifications à l'arrache, parce qu'elle me plaît, parce que j'ai envie, parce que mon voisin à dit c'est cool.
-> Un modèle de situation de crise pour qui ? Pour quoi ?
2. Des tonnes d'architectures différentes, en fonction de comment le graduate a souhaité l'implémenter. Pas/peu de solution pour implémenter des données.
-> MLOps appliquées à notre problématique + mon architecture HICSS
3. Beaucoup de travaux pour classifier des messages. Peu de travaux sur le traitement à l'échelle du mot.

% \begin{figure}[bp]
%   \centering
%   \begin{tikzpicture}
%     \fill[orange] circle(1);
%     \fill[lime] (1,1) rectangle ++(1,1);
%   \end{tikzpicture}
%   \caption{Exemple de figure}
%   \label{fig:ex}
% \end{figure}


%%% Local Variables:
%%% mode: latex
%%% TeX-master: "../ma-these"
%%% End:
