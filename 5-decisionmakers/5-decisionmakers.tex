\chapter*{System of processing systems: implementing crisis situation models using social media data}

% /?\ Chapitre dangereux : Ne pas se perdre dans des trucs que je ne sais pas faire (Interface IHM)

\section{Challenges for Information Systems in crisis response}

\section{Partie technique}
% Partie SI qui accueille l'agorithme précédent
%  /!\ Coeur de la partie : un SI pour la gestion de la gestion de crise AVEC DU ML
% Cela amene des problematiques de confiance dans le systeme est dans la gestion des donnees.
% Les donnees ne sont pas objectives : elles sont subjectives ; elles existent dans un contexte -> Comment on represente ça dans le SI ?
% Article HICSS avec detail des differentes parties et pourquoi elles sont importantes/quels sont les problemes
% Partie qui présente les informations que l'on peut retourner aux utilisateurs et comment elle doivent etre retourner
% Dire que l'infra c'est important, parce que les serveurs savent pas nager, et que des mecs s'en occupent (1 ref max).

% https://docs.google.com/document/d/1sP6TsuPSCSgoHZKEM7fqZ8BXZccv3JS8TobRJXEHD7Q/edit  Literature ISCRAM 
Section qui repond à la question : Comment met on en oeuvre concretement la solution apportee precedemment ?
Notre solution est un systeme d'aide à la decision sense fonctionner durant des evenements de crise.
Les systemes d'aide a la decision viennent avec des contraintes supplementaires par rapport du fait de leurs systemes de recommendations.
Il faut maintenir un niveau acceptable de qualite pour les recommendations.

En plus de cela, il est important de prendre en compte les contraintes liees a la gestion de crise, à savoir :
\begin{itemize}
    \item Fortes contraintes sur les ressources (humaines particulierement)
    \item La solution finale doit etre cheap
\end{itemize}

Liste de ressources pour alimenter cette partie :
\begin{itemize}
    \item Introducing MLOps
    \item Machine learning Engineering
\end{itemize}


\section{Partie design}
% Ouverture vers les techno de la communication 
% Faire une ouverture en présentant les possibilités, pas en donnant la solution
% Rappeler les demons de Endsley et que c'est important qu'un systeme soit developper POUR son utilisateur

Partie finale de la these.
Elle repond a la question : Comment est-ce qu'on fait pour delivrer les resultats de la these au preneur de decision ?
Pour ce faire, on propose de s'interesser à l'UI et l'UX du SI, et notamment la COP qui sera proposee au decision maker.
Plus precisement, comment est-ce que les suggestions faites par le systemes permettent de prévenir les daemons de la SA mentiones par Endlsey.
Liste des ressources utiles pour alimenter cette partie :
\begin{itemize}
    \item https://pair.withgoogle.com/guidebook/chapters
    \item https://pair.withgoogle.com/guidebook/patterns/how-do-i-get-started
    \item Les SA daemons de Endlsey
\end{itemize}


%%% Local Variables:
%%% mode: latex
%%% TeX-master: "../ma-these"
%%% End:
