\chapter{System of processing systems: implementing crisis situation models using social media data}

% /?\ Chapitre dangereux : Ne pas se perdre dans des trucs que je ne sais pas faire (Interface IHM)

% Partie SI qui accueille l'agorithme précédent
%  /!\ Coeur de la partie : un SI pour la gestion de la gestion de crise AVEC DU ML
% Cela amene des problematiques de confiance dans le systeme est dans la gestion des donnees.
% Les donnees ne sont pas objectives : elles sont subjectives ; elles existent dans un contexte -> Comment on represente ça dans le SI ?
% Article HICSS avec detail des differentes parties et pourquoi elles sont importantes/quels sont les problemes
% Partie qui présente les informations que l'on peut retourner aux utilisateurs et comment elle doivent etre retourner
% Dire que l'infra c'est important, parce que les serveurs savent pas nager, et que des mecs s'en occupent (1 ref max).

% From a sociotechnical perspective, information systems are composed by four components: task, people, structure (or roles), and technology.
% Information systems can be defined as an integration of components for collection, storage and processing of data of which the data is used
% to provide information, contribute to knowledge as well as digital products that facilitate decision making."

% https://docs.google.com/document/d/1sP6TsuPSCSgoHZKEM7fqZ8BXZccv3JS8TobRJXEHD7Q/edit  Literature ISCRAM 

Section qui repond à la question : Comment met on en oeuvre concretement la solution apportee precedemment ?
Notre solution est un systeme d'aide à la decision sense fonctionner durant des evenements de crise.
Les systemes d'aide a la decision viennent avec des contraintes supplementaires par rapport du fait de leurs systemes de recommendations.
Il faut maintenir un niveau acceptable de qualite pour les recommendations.

En plus de cela, il est important de prendre en compte les contraintes liees a la gestion de crise, à savoir :
\begin{itemize}
    \item Fortes contraintes sur les ressources (humaines particulierement)
    \item La solution finale doit etre cheap
\end{itemize}

Liste de ressources pour alimenter cette partie :
\begin{itemize}
    \item Introducing MLOps
    \item Machine learning Engineering
\end{itemize}

\section{Challenges for Information Systems in crisis response}
All the components listes and described in the rest of this section are primordial for the success of this system.
However, as per the literature review, few attention span is spent on the distribution of the information (last phase).
This part is however crucial to the success of such system, yet required a scientific multi disciplinary approach, that is hard to create among researchers.

Specific chalegnes that arise with crisis management and crisis response.
Reminder of stakes mentioned in the introduction + other issues that arise.

Also, there are, in fact two systems in the information system that is going to be mentioned here.


\section{The Data system}
Here we are back at the information system (IS) mentioned in the introduction.
As a reminder "An information system (IS) is a formal, sociotechnical, organizational system designed to collect, process, store, and distribute information." (Wikipedia)
Considering the Information as in the Data-Information-Knowledge framework, Information is obtained from Data.
Thus, an Information System has to be a system of systems:
\begin{itemize}
    \item A Data system
    \item An Information system
    \item A Knowledge system — forget about that one here
\end{itemize}
That have a solid INformation system, you first need a Data system.
So a system that collect, process, store and distribute data -> the Data system.
An Information system, that collect, process, store and distribute information -> the information system.
Both live in harmony, one is useless without the another.
The catch is that one deliver value to the end uer (the decision makers), the IS, but not the data system.

So, to get a good IS, it is important to first understand and design correctly the data system.

\subsection{Data Collection}
Multiple sources of data:
\begin{itemize}
    \item Reports
    \item Phone calls
    \item Social media
    \item Sensors (IoT: water levels, houses)
    \item Drones footages
\end{itemize}
Different types of data.

=> Build a system that is able to collect data from all these sources at the same time, in am organized, extensible way.
Use of a Message broker/Stream processor as a datalake to hold the data, waiting for
Topics corresponding to the different soruces of data.

\subsection{Data Process}
Once the data are collected, they have to be cleaned up prior to storage.
Addition of metadata to provide further context.
Normalization of data (encoding of textual data for instance)

\subsection{Data Storage}
Storage of data in an relational database, to ease future use of the data.
This allow to create a datastore with structured data.
This is the heart of the system and from where all the value come from.
The caching of some data, that will be reused often is also of interest in this part.

There are several questions taht arise concerning the political governance of this aspect (privacy, authorities in charge etc.)

\subsection{Data Distribution}
Distribution of data through several APIs available, where the different services can request data according to their need.
These services deliver the data upon request of the next system that will talk about: the information system.

\section{The Information system}
The Information system uses the Data collected and stored in the previous system.
This system process the data through clustering, aggregation etc to deliver value, that is in our case, relevant insights for the decision maker.

\subsection{Information collection}
Interface of the system that asks for the data.
This part is aligned with the distribution part of the Data System.
Input of the required data to get the processing working.

\subsection{Information processing}
The data collected are then processed to create information.
Also, previous information can be reused to create new information.

This step also concerns the creation of structures between information (relations, cluster, layers...)

\subsection{Information storage}
Information created need to be stored just as data.
The catch here is the creation of valuable metadata to store along information (confidence in that information, possible related information etc.).
Just like data, this part mentions the caching of certain information that will be reused often.

\subsection{Information distribution}
This part focuses on the delivery of the different information to the end user.
% Ouverture vers les techno de la communication 
% Faire une ouverture en présentant les possibilités, pas en donnant la solution
% Rappeler les demons de Endsley et que c'est important qu'un systeme soit developper POUR son utilisateur

The end goal of the system.
This step is really important, as it is where the value is delivered.
The distribution is a bottleneck of the system.
To get it done correctly, the right information has to be delivered at the right person at the right time.
If this part is not done correctly, all previous work is useless.
So first, design distribution correctly, then build the system.

Get back to Endlsey's daemons that highlight the importance of that topic.

Opening on
\begin{itemize}
    \item Caroline's work
    \item HMI design
\end{itemize}

The design of the elements of this part of the system is critical.
A reflexion on the design of elements such as COP is important , but out of the scope of this manuscript.
However, critical research
Pour ce faire, on propose de s'interesser à l'UI et l'UX du SI, et notamment la COP qui sera proposee au decision maker.
Plus precisement, comment est-ce que les suggestions faites par le systemes permettent de prévenir les daemons de la SA mentiones par Endlsey.
Liste des ressources utiles pour alimenter cette partie :
\begin{itemize}
    \item https://pair.withgoogle.com/guidebook/chapters
    \item https://pair.withgoogle.com/guidebook/patterns/how-do-i-get-started
    \item Les SA daemons de Endlsey
\end{itemize}

%%% Local Variables:
%%% mode: latex
%%% TeX-master: "../ma-these"
%%% End:
