\chapter*{Decision Makers}

\section{Partie technique}
Section qui repond à la question : Comment met on en oeuvre concretement la solution apportee precedemment ?
Notre solution est un systeme d'aide à la decision sense fonctionner durant des evenements de crise.
Les systemes d'aide a la decision viennent avec des contraintes supplementaires par rapport du fait de leurs systemes de recommendations.
Il faut maintenir un niveau acceptable de qualite pour les recommendations.

En plus de cela, il est important de prendre en compte les contraintes liees a la gestion de crise, à savoir :
\begin{itemize}
    \item Fortes contraintes sur les ressources (humaines particulierement)
    \item La solution finale doit etre cheap
\end{itemize}

Liste de ressources pour alimenter cette partie :
\begin{itemize}
    \item Introducing MLOps
    \item Machine learning Engineering
\end{itemize}


\section{Partie design}
Partie finale de la these.
Elle repond a la question : Comment est-ce qu'on fait pour delivrer les resultats de la these au preneur de decision ?
Pour ce faire, on propose de s'interesser à l'UI et l'UX du SI, et notamment la COP qui sera proposee au decision maker.
Plus precisement, comment est-ce que les suggestions faites par le systemes permettent de prévenir les daemons de la SA mentiones par Endlsey.
Liste des ressources utiles pour alimenter cette partie :
\begin{itemize}
    \item https://pair.withgoogle.com/guidebook/chapters
    \item https://pair.withgoogle.com/guidebook/patterns/how-do-i-get-started
    \item Les SA daemons de Endlsey
\end{itemize}


%%% Local Variables:
%%% mode: latex
%%% TeX-master: "../ma-these"
%%% End:
