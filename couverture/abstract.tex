% le résumé (en 4e de couverture)
\cleardoublepage
\pagestyle{empty}
\null
\newpage
\abstract[0.6]%
{french}{Résumé}%
{Conception d'un système de traitement des médias sociaux en réponse de crise : extraction, gestion et distribution des informations pertinentes pour les décideurs.}%
{
  Nos sociétés ont toujours été ponctuées de situations de crises, mais la complexité croissante de ces
  événements exige une amélioration constante des méthodologies et des outils employés lors de la réponse.
  L'établissement d'une conscience de la situation commune à tous les acteurs impliqués est l'une de ces améliorations
  potentielles.
  Cependant, cette axe d'amélioration souffre de difficultés liées au manque de ressources à allouer à cette tâche.
  L'automatisation d'une partie des tâches pour supporter le personnel en charge de cet aspect, est donc une opportunité de recherche.
  Cette opportunité est également favorisée par le développement des médias sociaux en tant que sources de données massives.
  Simultanément, le domaine de l'intelligence artificielle a été radicalement modifié par
  le développement de nouveaux outils et de nouvelles méthodes, permettant la recherche
  d'informations complexes au sein de données textuelles.
  À la croisée de ces trois opportunités conjugués, cette thèse explore la question suivante :
  Comment concevoir un système d'information capable de gérer et de fournir automatiquement
  des informations pertinentes extraites des données des médias sociaux ?

  Une approche en trois temps est proposée. Premièrement, il s'agit de comprendre
  quelles sont les informations pertinentes lors de la phase de réponse à une crise pour les preneurs de décision.
  Deuxièmement, une fois les informations pertinentes identifiées, un nouveau module d'intelligence
  artificielle dédié extrait les éléments pertinents à partir des données disponibles sur les médias sociaux.
  Ces informations sont alors intégrées dans un modèle de situation de crise, permettant
  de les organiser automatiquement avec le reste du contexte.
  La troisième et dernière partie discute de l'organisation des données et
  des informations au sein d'un système d'aide à la décision pour la gestion de crise.
  Cette discussion s'intèresse particulièrement à la question de la bonne gestion et de
  la distribution de ces informations auprès des décideurs.
  Cette recherche a été menée dans un contexte international : le projet français ANR
  MACIV, une collaboration entre IMT Mines Albi et Penn State University et en relation
  étroite avec des praticiens français et américains.

  \keywords{Mots-clés :}{Gestion de crise, Apprentissage Machine, Traitement Automatique du Langage, Connaissance de la Situation, Système d'information}
}
{american}{Abstract}%
{Design of a social media processing system for crisis response: extraction, management and delivery of relevant information for decision makers}%
{
  Our societies have always been punctuated by crises, but the increasing complexity of these events requires a constant improvement of the methodologies and tools used in the response.
  Establishing a common situational awareness among all actors involved is one of these potential improvements.
  However, challenges arise due to the lack of available resources to allocate to this task during crisis response.
  The automation of certain tasks to support teams' dedicated actionable information collection, therefore, represents a research opportunity.
  This opportunity is also enabled by the expansion of social media as big data sources.
  At the same time, the field of artificial intelligence has been radically changed by the development of new tools and methods, allowing the retrieval of complex information within textual data.
  At the crossroads of these three opportunities, this dissertation explores the following question:
  How to design an information system capable of managing and automatically providing relevant information extracted from social media data?

  A threefold approach is proposed.
  The first part aims at understanding what information is relevant in the crisis response phase for decision-makers
  Second, once the relevant information is identified, a new, dedicated artificial intelligence
  module extracts the relevant elements from the data available on social media.
  This information is then integrated into a crisis model, allowing to automatically
  pair it with associated information available in the context.
  The third and last part discusses the organization of data and information within a decision support system for crisis management.
  This discussion is particularly interested in designing a system that can achieve proper management and distribution of information to decision-makers.
  This research was conducted in an interdisciplinary context: the French ANR MACIV project, a collaboration
  between IMT Mines Albi and Penn State University and in close relationship with French and American practitioners.

  \keywords{Keywords:}{Crisis Management, Machine Learning, Natural Language Processing, Situation Awareness, Information System}
}

%%% Local Variables:
%%% mode: latex
%%% TeX-master: "../ma-these"
%%% End:
