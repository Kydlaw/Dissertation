\chapter{Context and problematic}

\section{Crisis management domain}
\subsection{A definition?}
Defining the concept of crisis provides a hint of the challenges that lie within this domain.
Bien que le terme est largement adopte dans le language courant, il est paradoxalement difficile d'en fournir une definition scientifique precise et definitive.
Le terme est employe aussi pour bien pour une krash financier, qu'une catastrophe naturelle ou humanitaire.
De nombreux chercheurs ont pourant essaye de d'arreter une definition a ce concept flou.
\cite{lagadecGESTIONCRISES1994} se rendait deja compte en 1994 que de tres nombreuses definitions existaient.
Il dresse alors un inventaire des différentes approches employees pour cerner ce concept.
On retrouve ainsi des taxonomies comme celle propose par \cite{rosenthalCrisisDecisionMakingNetherlands1986}, qui, souhaitant une plus large definition
de "crise", se sont interesse aux differents types de crises. Ils proposerent alors la taxonomie suivante :
\begin{itemize}
    \item The "unimaginable" crisis, requiring that we really question the unthinkable (it does not appear very frequently).
    \item The "neglected" crisis.
    \item The "almost inevitable" crisis, in spite of a preventive action.
    \item The "compulsive" crisis, which results from inadequate management.
    \item The crisis sought by some, internal or external, actors.
    \item The crisis deeply desired, by all parties.
\end{itemize}
De maniere quasi similaire, \cite{mitroffStructureManmadeOrganizational1988} proposerent une classification des crises en fonction de caracteristics intrinseques.
Les auteurs utilisent une matrice 2D pour classer les différents types d'événements.
L'un des composante oppose l'origine de l'evenement (interne ou externe) tandis que l'autre axe oppose les aspects "sociaux" ou "techniques".
On retrouve ainsi les faillites d'entreprises dans le cadran interne/technique, les attaques terroristes dans le cadran externe/humaine ou encore les catastrophes
naturelles dans le cadran externe/techniques.
Des définitions succintes ont egalement ete proposes pour definir le terme en lui meme.
\cite{hermannIssuesStudyInternational1972} proposa ainsi que "une crise est une situation qui menace les buts essentiels des unités de prise de décision,
réduit le laps de temps disponible pour la prise de décision, et dont l'occurrence surprend les responsables".
Plus qu'une simple situation, \cite{rosenthalCrisisDecisionMakingNetherlands1986} prefera insister sur la notion de prise de decision cruciale.
Ainsi, "Une crise est une menace sérieuse affectant les structures de base ou les valeurs et normes fondamentales d'un système social,
qui – en situation de forte pression et haute incertitude – nécessite la prise de décisions cruciales."
Mais les crises sont egalement des periodes d'incertitude et d'égarement des organisations, ou les regles et les processus sont brouillés.
\cite{lagadecGESTIONCRISES1994} abandonna quelque peut la notion de succinte en proposant une definition plus ambitieuse en
adoptant un point de vue plus haut niveau. Ainsi une crise est "une situation où de multiples organisations, aux prises avec des problèmes critiques,
soumises à de fortes pressions externes, d'âpres tensions internes, se trouvent projetées brutalement et pour une longue durée sur le devant de la scène;
projetées aussi les unes contre les autres... le tout dans une société de communication de masse, c'est-à-dire "en direct", avec l'assurance de faire
la «une» des informations radiodiffusées, télévisées, écrites, sur longue période."

From the definitions given above, one can see the difficulty of defining the concept of crisis, as it is so diverse.
Crisis situations, although they appear to be a constant in our societies, seem to be out of reach due to the lack of regularity in the concept.
In the end, crises seem to be the demons living in the dark face of our societies.
Invisible and seemingly out of reach, we are only witnesses of their sudden and brutal irruptions in the visible phase of our world.
In fact, these irruptions invariably result in an eruption of chaos.
This metaphorical representation translates my personall vision of what a crisis is and the inherent complexity of the definition of this concept.
But while describing those "demons" seems an incredibly difficult endeavour, the irruptions themselves and their consequences possess common points.
Theses common points are the caracteristics discussed in the following.

% \begin{table}[bp]
%   \centering
%   \begin{tabular}{rrr}
%     & A & B \\
%     \toprule
%     Exemple 1 & 10 & 35 \\
%     Exemple 2 & 12 & 33 \\
%     \bottomrule
%   \end{tabular}
%   \caption{Un exemple de tableau}
%   \label{tab:ex}
% \end{table}

%%% Local Variables:
%%% mode: latex
%%% TeX-master: "../ma-these"
%%% End:
