\chapter*{Conclusion and Perspectives}

The response to a crisis is strongly influenced by the ability to make the right decision at the right time.
However, the reality of the crisis makes this objective more remote and often forces compromises.
In the hope of reducing the gap between reality and the objective, many actors have sought to improve information feedback to decision-makers.
This work is part of that effort, seeking to serve content posted on social media to those who need it.
However, many challenges emerge.
Social media is not the only source of data to feed decision-makers' situational awareness.
Faced with this influx of data and information, it is essential to limit as much as possible the extra and limit the mental load attributed to this task.
This observation only increases the status of the information system (IS) in crisis management.
The latter then becomes responsible for collecting and processing the data and providing the most relevant information to decision-makers.
This vision has guided all of the previous work.
The main contributions are summarized in the following.

\section*{Summary}
The first chapter of this dissertation presented its scientific background composed of three domains: (i) crisis management, (ii) social media, and (iii) natural language processing.
This Ph.D. research works was conducted at the intersection of these domains and identified the principal research question that guided the rest of the document: \emph{How to design an information system for crisis response that can automatically manage and deliver actionable information from social media data?}

The second chapter provided an overview of previous works on the different aspects of the research question.
It highlights the coexistence of two fields around these issues: information sciences and social sciences.
Each field explores the space of the problem and meets at certain intersections.
This part also shows the important interest of the scientific community for the automatic processing of social media content and the work to implement it in crisis management.

In this context, the third chapter asks what information decision-makers need from social media.
Once this information is identified, it needs to be represented in a computerized way to automate its organization and collection.
The information science community has proposed numerous computer representations, or information models, for crisis management.
The last part of this chapter presented an information model for the information available on social media for decision-makers.
This model is obtained by intersecting the needs identified in the first part with the relevant information models identified in the second part.

The fourth chapter builds on the earlier information model to propose a method for automatically identifying this information.
This method builds on previous work that seeks to differentiate between messages related to the event and those that are not.
The processing among the resulting messages is similar to recognizing named entities, with the difference that the entities are not named but correspond to actionable information.
This identification is performed using a machine learning method.
Training the machine learning model in a supervised fashion raises challenges in the context of crisis management.
Hence, a semi-supervised approach is preferred for training the model.
The latter is based on the word embeddings created by a language model such as BERT.
The dimension of the vectors provided by BERT is then adopted to identify the clusters of terms present in the dataset provided as input to the algorithm.
Some of these clusters contain terms that the user has previously labeled according to the actionable information of the model.
The labels of these terms are then propagated to all the other unlabeled terms present in the cluster.
In this way, the terms labeled by the user and their synonyms are highlighted in the incoming message flow.

Finally, the fifth and last chapter considers the IS as a whole.
In particular, it asks about the role and issues of integrating the data processing method into an IS for crisis management.
After studying the literature on the associated issues in the first part, it is concluded that such an IS is composed of two parts.
One part is dedicated to managing data, which are used in part to feed the AI at the heart of the IS.
A second part is responsible for information management and considers user input and output.
Another aspect of the proposed dichotomy is the distinction between data and information storage.
The first case is carried out with the help of a traditional relational database.
Information storage is based on the association of an information model, such as those mentioned above, and a graph-oriented database.
The graph database allows defining instances of the information present in the model and semantic relations between the different cases.
This approach promises many evolutions in processing information exchanged during a crisis.

\section*{Perspectives}
The work presented in this document covers a relatively large study area by trying to mix human and technical around this scientific challenge.
However, at the dawn of this document, more questions remain unanswered than answers have been given.
This field of research remains rich and challenging, both scientifically and technically.
The following parts present potential research directions that are currently foreseen.

\subsection*{Short terms}
The problem of misinformation and fakes was mentioned in Chapter 1 of this dissertation.
Although excluded in this work, this issue is more than ever decisive for the success of social media-based modules.
\textcite{batardIntegrerContributionsCitoyennes2021} refers to both beneficial and detrimental citizen initiatives in his work.
Some citizens may actively use social media to share ideas or ideologies detrimental to crisis management.
The results obtained around this issue must now be integrated into systems.
Their objectives would be to detect and fight more upstream against potential trends that could harm the response.
This problem remains a significant issue for many practitioners who regularly mention it as a deterrent to adopting such systems.
This integration would reassure them and thus facilitate adoption and experimentation.

Additionally, it would be interesting to consolidate the several proposals made by the crisis informatic community over the years.
These proposals should now be improved according to the practitioners' feedback.
Their feedback will also be an opportunity to discuss the functioning of the information systems as a whole.
In parallel to these improvements, discussions should take place on new ways of acquiring or interpreting data to feed the IS.
As far as interpretation is concerned, a first step could be to refine the previously proposed classifications.
This would allow the identification of the most relevant information for decision-makers.

\subsection*{Mid-long terms}
A better understanding of the expectations and needs of practitioners, especially when dealing with unique situations, would also be very beneficial.
A large volume of publications takes a top-down approach, proposing solutions directly to users.
However, many of the observations made by the practitioners reveal and reflect their difficulties and needs.
Better integration and collaboration between domains would necessarily lead to better results for crisis management.
Practitioners who are among the early adopters are key players.
Their feedback and testimonies save precious time searching for appropriate solutions to the issue at hand.

This document also mentioned the numerous data sources that can support crisis management.
Messages posted on social media and photos or videos are all information accessible from social media.
However, other sources have been mentioned, such as video streams from drones or readings from various sensors.
The collection and automatic interpretation of these data is currently the focus of scientific attention.
This results in a large amount of information being generated through the interpretation of the data.
This heterogeneity is an asset, and many developments are to be expected thanks to it.

A future challenge will be to homogenize the available information to exploit its full potential.
Such homogenization is made possible by a standard information model that allows the standardization of information obtained from different sources.
Two direct consequences are expected.
First, using several sources allows for different incidents or different geographical areas.
Thus, someone can report an event in progress in an area not covered by a camera or vice versa.
This consequence corresponds to an extension of the information or the \emph{horizontal} scaling of the coverage.

This heterogeneity also allows for \emph{vertical} scaling.
Having different data sources covering the same event can provide additional information than a single source.
For example, someone can warn via social media that an event is taking place, and a camera on the spot can better understand the context and type of event concerned.
This application thus corresponds to an \emph{enrichment} in information.
This enrichment can also take place on the same information.
For example, a user indicates an event has taken place and mentions a location.
At the same time, a surveillance camera also detects an event at this exact location.
The concordance of these two pieces of information reduces the uncertainty of some information.
Providing an indicator of certainty in the information proposed by the system would also reduce the mental load on decision-makers.

Finally, creating vector representations to encode the semantics of the data presented in Chapter 4 may see other applications.
Different sources or different types of data could have their data encoded similarly.
The resulting set of representations would then be normalized into a common framework.
The data within this framework could be processed using an approach similar to Chapter 4 to identify the different concepts at play during an event.
This would allow the emergence of information and concepts adapted to the current event rather than the generic and fixed ones of a pre-established metamodel.

% Fred: Je pense pour finir que ce travail pour la question de la représentation abstraite de la connaissance.
% Tu pourrais conclure sur la volonté d'exploiter la représentation multi-dimensionnelle du savoir que tu obtiens et de son exploitation par delà le seul mapping du MM.
% À discuter si tu veux 

%%% Local Variables:
%%% mode: latex
%%% TeX-master: "../ma-these"
%%% End:
