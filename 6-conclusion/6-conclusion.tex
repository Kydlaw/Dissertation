\chapter*{Conclusion}

\section{Discussion}


\section{Conclusion}
Il est evident qu'etre capable de capturer la dimension sociale permet de mieux apprehender certains problemes.
La gestion de crise est evidement tres fortement inluencee par cette capacite de comprendre la population et son rensentit au cours de l'evenement.
Cependant, l'usage des medias sociaux, s'il est possible comme venons de le montrer, reste relativement limité, des raisons différentes de raisons techniques.

Tout d'abord, les plateformes limitent les données qu'il est possible d'extraire.
Cette décision fait notamment suite au scandale de Cambridge Analytica et l'usage qu'il a été fait des données des utilisateurs de Facebook dans le cadre la compagne d'elections presidentielles Americaines de 2017.
Conscientes du pouvoir que confere la comprehension des informations que procurent la connaissance d'un reseau social, ces medias ont preferes fermer leurs acces.
Cette limitation est politique, et il revient aux états, en partenariat avec les plateformes, de decider de l'acces raisonnable aux donnees des utilisateurs, dans differents contextes.


%%% Local Variables:
%%% mode: latex
%%% TeX-master: "../ma-these"
%%% End:
