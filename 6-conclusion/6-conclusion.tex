\chapter*{Conclusion}

\section{Discussion}


\section{Conclusion}
Il est evident qu'etre capable de capturer la dimension sociale permet de mieux apprehender certains problemes.
La gestion de crise est evidement tres fortement inluencee par cette capacite de comprendre la population et son rensentit au cours de l'evenement.
Cependant, l'usage des medias sociaux, s'il est possible comme venons de le montrer, reste relativement limité, des raisons différentes de raisons techniques.

Tout d'abord, les plateformes limitent les données qu'il est possible d'extraire.
Cette décision fait notamment suite au scandale de Cambridge Analytica et l'usage qu'il a été fait des données des utilisateurs de Facebook dans le cadre la compagne d'elections presidentielles Americaines de 2017.
Conscientes du pouvoir que confere la comprehension des informations que procurent la connaissance d'un reseau social, ces medias ont preferes fermer leurs acces.
Cette limitation est politique, et il revient aux états, en partenariat avec les plateformes, de decider de l'acces raisonnable aux donnees des utilisateurs, dans differents contextes.

À l'issu de ce manuscript, le lecteur a pu se rendre compte de la complexité de la problématique initiale.
Ce problème, multi dimensionnel, s'étend au-delà du domaine de science des systèmes d'information.
À ce stade, il apparait évident que seule une approche multi disciplinaire et multi acteurs peut
fournir une réponse satisfaisante au problème de l'intégration des informations issues des
médias sociaux dans le processus de prise de décision durant la réponse à la crise.
La coopération entre tant d'acteurs est à la fois cruciale pour la réussite de ce projet,
mais également sans plus grand frein.
Il est difficile d'établir des conversations entre les différentes communautés, à moins
que chaque partie accepte ne fasse preuve de curiosité et acceuille le point de vue de
l'autre communauté de recherche.
La bureaucratie et les processus mise en place autour des, ou par, les practitioners
ralentie également considérablement le progrès.
Les acteurs les moins intéressés, même s'ils réalisent à l'issu les avantages de la démarche,
peuvent trainer initialement des pieds, ce qui entraine perte de temps, d'énergie et frustrations.
Ainsi, il est important de capitaliser sur les practioners qui sont intéressés fortement
(earlys adopters), de les accompagner et de maintenir une relation de confiance avec eux.


% TODO Matrice SWOT de la résolution de ce problème.
Les freins:
\begin{itemize}
    \item Problème de confiance des opérationnel envers les médias sociaux. -> Ce frein sera levé en ayant des succès opérationnels.
          L'accompagnement des earlys adopters est critique pour cette phase.
    \item Peu d'intéropérabilité entre les différents systèmes
    \item Système souvent complexe etc.
\end{itemize}

Les opportunités identifiés

Les menaces:
\begin{itemize}
    \item Perte d'intérêt des chercheurs qui courrent après le new shiny thing -> tout le travail sera donc inutile, car perte de motivation des practitioners, qui est indispensable.
    \item
\end{itemize}

% TODO Mettre en avant les opportunités offertes par cette architectures
% TODO Ajouter le futur work, ce qu'il est possible faire

\subsection{Architecture composite de traitements}
Composition de traitements à l'échelle macro et micro.
=> Traitements à l'échelle méta de l'événement
=> Metadonnées

\subsubsection{Macro traitement}

\subsubsection{Micro traitement}

\begin{itemize}
    \item Utilisation du pool de données pour inférer davantage d'informations
\end{itemize}


%%% Local Variables:
%%% mode: latex
%%% TeX-master: "../ma-these"
%%% End:
