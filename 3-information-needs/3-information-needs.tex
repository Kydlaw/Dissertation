\chapter{Crisis situation models that serve social media operators}

%% Jonction entre le : "quel modele je peux constuire avec les donnees issues de medias sociaux" et "de quoi on besoin les gens qui utilisent les media sociaux en gestion de crise"
\section{Who process social media during crisis response?}
- 911 calls takers, social media experts in the US (meetings with the operators of the call centers)
- Social media operators in France (MACIV exercices)

\section{Needs identified}
\subsection{Situation awareness}
- Situation awareness according to Endsley
- Situationa awareness adapted to NLP for crisis response

\subsection{Actionable_information}
- Jess Kropczynksi's interviews. 6W's
- Zahra's survey and interviews
- Others?

\section{Intersection with crisis situation models}
- Intersection with the high level crisis situation models presented in the LR and the "feedbacks from the field".
- My article ISCRAM 2021 (for the need to ackownelge the users' needs in the design of social media processing systems) and HICSS 2021 (for the classification of information that we are looking at).

=> Classes that are relevant for the end users/classes we will search for.

\subsection{Conclusion}
- We figured what information we were looking for. Now, how do we extract those information? How do we implement the associated classes of the metamodel?
- Update the table in the LR with the identified needs


%%% Local Variables:
%%% mode: latex
%%% TeX-master: "../ma-these.tex"
%%% End:
