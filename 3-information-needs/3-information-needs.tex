\chapter{Crisis situation models that serve social media operators}


\section*{Introduction}
% Jonction entre le : "quel modele je peux constuire avec les donnees issues de medias sociaux" et "de quoi on besoin les gens qui utilisent les media sociaux en gestion de crise" %%
Reference to chap. 1 and chap.2 to emphasized on the collaboration.
Collaboration happens during all the phases of the crisis management cycle.
However the focus will be on the response phase.

Coordination of the response requires collaboration.
Collaboration requires information sharing.
Yet, information sharing is problematic as actors need a common vocabulary to talk about things that everybody can understand.
Also, these information need to be served in a common way to all the actors (come back to the COP).
These measures facilitate the decision making of all the actors.
Finally, decisions made by each actors need to be passed to the other actors.

What are the information that people looking at social media need to share with decison makers?
Thus, this chapter focuses on two aspects:
\begin{itemize}
    \item Who look at social media and what are they looking for?
    \item Which information model is it possible to build using social media data?
\end{itemize}

\section{Who process social media during crisis response?}
Crisis management involve multiple actors (see chap. 1).
Thus, not all member of the organization are dealing with social media processing.
This section aims at clarifing the first part of the original problematic: who are the person dedicated to social media content processing during emergency events?

The argumentation in this section is mostly anecdotical, with the meetings with the US 911 and MACIV exercices.
The MACIV project was really insightful in this regard.
Multiple occasions to witness the organization of crisis management, while it wasn't the exact focus of the project (mostly information focused).

For the US: call centers (handling it as the regular calls, with people checking social media for information).
The information is then sent to the dispatchers or to the crisis cell if there is one.
In France: a third party watch social media (VISOV association) and their results are then consolidated in the emergency management centers upon request

Conclusion: Here are the different profils identified that are facing social media during an emergency event.
2 things: they use it to obtain information from social media for 2 purposes:
\begin{itemize}
    \item Help organize the response on the field (information consolidation)
    \item Orient their communication to the public (information dissemination)
\end{itemize}
Also, important to mention that from my perspective, the population of call centers/emergency centers that are interesed in social media is not the majority of them.
So there is room for improvement here.

\section{Information needs identified}
Qu'est-ce que les profils cherchent concrètement lorsqu'ils regardent les médiaux sociaux ?

- Partie terrain (Jess - 6W)
- Partie "+ remote" (Audrey - Robin ...)

\subsection{Situation awareness}
- Situation awareness according to Endsley
- Situationa awareness adapted to NLP for crisis response

\subsection{Actionable information}
- Jess Kropczynksi's interviews. 6W's
- Zahra's survey and interviews
- Others?

=> Voici les informations dont ils ont besoin (definitions SA et AInf ici?)

\section{Intersection with crisis situation models}
- Intersection with the high level crisis situation models presented in the LR and the "feedbacks from the field".
- My article ISCRAM 2021 (for the need to ackownelge the users' needs in the design of social media processing systems) and HICSS 2021 (for the classification of information that we are looking at).

Retour sur les models d'information pour le CM.
1. Filtrer les modeles qui n'incluent les utilisateurs precedement identifies dans leurs scopes
2. Identifier les classes utilises par ses modeles qui representent l'information dont les utilisateurs ont besoin

=> Classes that are relevant for the end users/classes we will search for.

\section{Conclusion}
- We figured what information we were looking for. Now, how do we extract those information? How do we implement the associated classes of the metamodel?
- Update the table in the LR with the identified needs

=> Voila le model des informations qu'on peut esperer pour la crise pour les personnes en des medias sociaux

Next chapter deals with the ways to extract the information identified in this chapter.

\section{Who process social media during crisis response?}
\begin{itemize}
    \item 911 calls takers, social media experts in the US (meetings with the operators of the call centers)
    \item Social media operators in France (MACIV exercices)
\end{itemize}

\section{Needs identified}
\subsection{Situation awareness}
\begin{itemize}
    \item Situation awareness according to Endsley
    \item Situationa awareness adapted to NLP for crisis response
\end{itemize}

\subsection{Actionable information}
\begin{itemize}
    \item Jess Kropczynksi's interviews. 6W's
    \item Zahra's survey and interviews
    \item Others?
\end{itemize}

\section{Intersection with crisis situation models}
Intersection with the high level crisis situation models presented in the LR and the "feedbacks from the field".
My article ISCRAM 2021 (for the need to ackownelge the users' needs in the design of social media processing systems) and HICSS 2021 (for the classification of information that we are looking at).

=> Classes that are relevant for the end users/classes we will search for.

\subsection{Conclusion}
We figured what information we were looking for. Now, how do we extract those information? How do we implement the associated classes of the metamodel?
Update the table in the LR with the identified needs


%%% Local Variables:
%%% mode: latex
%%% TeX-master: "../ma-these.tex"
%%% End:
